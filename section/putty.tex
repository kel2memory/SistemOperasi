%Kelompok Memory Allocation (2)
%Arjun Yuda Firwanda
%Dezha Aidil Martha
%Dwi Septiani Tsaniyah
%Muh.Rifky Prananda
%Yusuf Al-Qordhawi

Putty

\section {Membahas Tentang "Putty"}

\subsection {A. Memahami Putty}

Putty adalah program open source yang dapat digunakan untuk melewati protokol jaringan SSH (Secure Shel), Telnet dan Rlogin. Protokol ini dapat digunakan untuk remote pada komputer melalui jaringan, baik LAN (Local Area Network) atau internet. Program ini digunakan oleh pengguna komputer saat ini, sekarang untuk menghubungkan, mensimulasikan, dan masalah yang terkait dengan jaringan. Program ini tentu saja juga sebagai sebuah terowongan (enkapsulasi atau pembungkusan protokol) di dalam jaringan.
Protokol dapat digunakan untuk mengetahui jaringan atau menjalankan sesi jarak jauh di komputer.

\begin{enumerate}

	\item SSH (Secure Shell)
	Merupakan protokol jaringan yang memungkinkan pertukaran data melalui saluran aman antara dua perangkat jaringan.

	\item Rlogin
	Sistem yang memungkinkan kita dapat masuk dari satu sistem ke sistem lain tanpa kata sandi tambahan.

	\item Telnet
	Adalah jaringan telekomunikasi yang digunakan di Internet atau Local Area Network (LAN) untuk menyediakan fasilitas komunikasi berbasis teks yang menggunakan koneksi terminal virtual.

\end{enumerate}

\subsection Pengertian SSH (Secure Shell)

SSH adalah suatu kriptografi yang digunakan untuk mengkomunikasikan data yang ada pada perangkat jaringannya untuk membuatnya lebih aman lagi. Dalam konsep menggunkan SSH ini harus di dukung oleh suatu server atau di dalam computer. Pada akun SSH ini dirancang untuk digunakan sebagai pola piker Telnet dan shell pada jarak jauh yang tidak aman, yang mengirimkan suatu informasi terutama pada kata sandinya dalam bentuk yang sederhanan dan mudah untuk disadap. Enkripsi disediakan oleh SSH untuk memberikan suatu kerahasiaan dan integritas data melalui jaringan tidak aman seperti internet.
 
